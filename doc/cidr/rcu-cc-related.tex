%% -*- tex-main-file:"rcu-cc.tex" -*-

\section{Related work}
\seclabel{related}

In terms of concurrency control, one of the most important studies has been \cite{AgrawalCL87}. This modeling study shows that if the overhead of pessimistic two-phase locking can be comparable to the overhead of optimistic methods then the pessimistic one is superior. The same study shows that it is beneficial to abort transactions that are going to abort as soon as possible. That is corroborated by other studies as well, e.g. \cite{PortsG12}. We follow the findings, trying to detect conflicts early.

Many of the memory-optimized systems adopt lightweight optimistic concurrency control schemes that are suitable only for a small fraction of transactional workloads.
The designs can be categorized in three categories: non-partitioning- and partitioning-based systems and clustered solutions. 
Silo's  \cite{TuZKLM13} employs a light-weight optimistic concurrency control that performs validations at pre-commit. That, as we showed in \secref{eval}, performs well only in a limited set of workloads. 
Microsoft Hekaton \cite{Diaconu+13} employs similar multi-versioning CC \cite{LarsonBDFPZ11}. It is worth mentioning that Hekaton also uses a technique similar to the indirection map, which we also use. 

%% \ippo{Partitioning-based}
H-Store (and its commercial version, VoltDB) is a characteristic partitioning-based system \cite{Kallman+08}. H-Store physically partitions each database to as many instances as the number of available processors, and each processor executes each transaction in serial order without interruption.  
Problems raise when the system has to execute mutli-site transactions, transactions that touch data from two or more separate database instances. 
Hyper \cite{KemperN11} follows H-Store's single-threaded execution principle.  To scale up to multi-cores they employ the hardware transactional memory capabilities of the latest generation of processors \cite{LeisKN14}. 
DORA \cite{PandisJHA10} employs logical partitioning that avoids limitations of physically partitioned systems. But DORA uses Shore-MT's codebase \cite{JohnsonPHAF09}, which is a scalable but disk-optimized storage manager with significantly bloated codebase. Hence, its performance lacks in comparison with the memory-optimized proposals.
