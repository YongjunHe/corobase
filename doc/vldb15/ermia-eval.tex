%% -*- tex-main-file:"rcu-cc.tex" -*-

\section{Evaluation}
\seclabel{eval}
In this section, we compare the performance of ERMIA and Silo, a representative of the lightweight OCC camp, from two perspectives: (1) the impacts of concurrency control scheme on performance and (2) scalability of the underlying storage manager. The purpose of our evaluation is two-fold. First, we show that although recent OCC proposals for main-memory OLTP perform extremely well, they are only suitable for a small fraction of transaction workloads---short update intensive transactions---primarily due to limitations imposed by their (tightly-integrated) concurrency control mechanism. Second, we demonstrate that ERMIA is able to offer robust concurrency control, maintaining high performance even under contentious workloads without sacrificing features. At the same time, ERMIA is also able to match the performance of specialized OCC solutions when running their favorite workloads. We also show that because of carefully orchestrated communications and interactions in both the physical and logical levels, ERMIA scales well under various types of workloads.

\subsection{Experimental setup}
\seclabel{eval:setup}
% System
We used a 4-socket server with 6-core Intel Xeon E7-4807 processors, for a total of 24 physical cores and 48 hyperthreads. The machine has 64GB of RAM. All worker threads were pinned to a dedicated core and NUMA node to minimize context switch penalty and inter-socket communication costs. Log records are written to tmpfs asynchronously. We measure the performance three systems: Silo, ERMIA with SI (ERMIA-SI), and ERMIA with SSI (ERMIA-SSI). Next we introduce the benchmarks.

\subsection{Benchmarks}
We run the TPC-C and TPC-E benchmarks on all variants. In each run, we load data from scratch and run the benchmark for 40 seconds.


\textbf{TPCE-original.}
TPC-E is a new standard OLTP benchmark. Although TPC-C has been dominantly used to evaluate OLTP systems performance for the past few decades, TPC-E was designed as a more realistic OLTP benchmark with modern features. \tianzheng{cite HANA paper/TPC-E spec} TPC-E models brokerage firm activities and has more sophisticated schema model and transaction execution control. It is also known for higher read to write ratio than TPC-C.


%TPCE + contention
\textbf{TPCE-contention.}
Since TPC-E is not a heterogeneous workload, we introduce a new read-mostly analytic transaction that evaluates total assets of a random customer account group and inserts the analytic activity log into \textit{analytic\_history} table. Total assets of the accounts are computed by joining \textit{holding\_summary} and \textit{last\_trade} tables. The vast majority of contentions occur between the analytic transaction and \textit{trade-result} transaction. The parameters we set for TPC-E experiments are 5000 customers, 500 scale factor and 10 initial trading days (the initial trading day parameter was limited by our machine's memory capacity). The followings are our new workload mix with the analytic transaction: broker-volume (4.9\%), customer-position (8\%), market-feed (1\%), market-watch (13\%), security-detail (14\%), trade-lookup (8\%), trade-order (10.1\%), trade-result (10\%), trade-status (9\%), trade-update (2\%) and analytic (20\%). 


\textbf{TPCC-fixed.}
TPC-C is well-known for update-heavy workload and small transaction footprints. Also, it is well-partitionable workload; conflicts can be reduced even further by partitioning and single-threadinging on local partitions, setting the almost ideal environment for the lightweight OCC. The only source of contention under partitioned TPC-C is multi-partition transaction, however, its impact on CC is dampened by small transaction footprints, avoiding the majority of read-write conflicts. In TPCC-fixed, entire database is partitioned by warehouse id, all worker threads are given local warehouse at benchmarking initialization, and does not change the warehouse binding during runtime. The fraction of cross partition transaction is 1\%. 


\textbf{TPCC-random.}
In TPCC-random, we enforce worker threads to pick a partition randomly during runtime, following non-uniform distribution. The purpose of this modification is to bring reasonable amount of contentions.  

\subsection{TPC-E}
\seclabel{eval:tpce}

\labeledfigurewide{fig-tpce-robustness}{Total throughput (left); analytic transaction throughput (right) of TPCE-hybrid, varying contention degree.\tianzheng{need normalized bars to show difference.}}

% chart - Commit rates / Abort rates, varying contention degree. 
We first explore how ERMIA and Silo react to contention in heterogeneous workload. \figref{fig-tpce-robustness} (left) shows the throughput with 24 worker threads, varying contention degree when we run TPC-E with the analytic transaction (TPCE-hybrid). To vary contention degree, we adjust the size of a customer account group to be scanned by the analytic transaction from 1\% to 60\%. At the extremely low contention level(1\%), Silo maintains performance reasonably. However, it starts to lag behind ERMIA-SI soon. From the 5\%,  ERMIA-SI outperforms Silo at all contention degrees and the performance gap gets larger until 40\% degree. Interestingly, the performance gap does not increase anymore from 40\%. This is because virtually all analytic transactions are already aborted at 40\% degree. Thus larger query footprint did not lead to larger performance gap. If the analytic query takes more than 20\% in the workload mix, the performance gap will get even larger. In case of ERMIA-SSI, it falls behind the Silo by 26\% at 5\%, however, it starts to close the gap quickly and catches up with Silo at 20\%. ERMIA-SSI is always slower than ERMIA-SI due to its non-negligible serializability checking cost and larger per-tuple cache footprint for keeping transaction dependency information. 
The main cause for the result is that the contention in the workload imposed heavy pressure on the OCC protocol; OCC enforced queries to abort even if single tuple of their read-set is invalidated by updaters. Meanwhile, ERMIA endured the contention by protecting the queries from updaters effectively with snapshot isolation and distributed the contentions across multiple versions. 
we now focus on the throughput of the analytic transactions to see how well both systems support heterogeneous workloads. \figref{fig-tpce-robustness} (right) illustrates normalized query throughput from the previous experiment. Even at the 1\% contention, Silo produced much lower query commits than both ERMIA-SI and ERMIA-SSI. As query size gets larger, Silo's query throughput sharply drops because the larger query is  more vulnerable to updaters. Consequently, analytic throughput collapsed on Silo; almost zero query commits at 40\%. This explains that the massive query aborts critically affected the overall throughput in the previous experiment. Thus, we should be aware of that it is extremely difficult, or impossible, to run heterogeneous workload under OCC protocol, even if net throughput looks reasonable. In recap, OCC prefers update transaction extremely over read-intensive query and will lead to query starvation if workload is contentious. Meanwhile ERMIA provides balanced query/transaction performance. This experiment shows that it is discouraging to deal with emerging heterogeneous transaction workload with OCC, putting SI as a promising alternative CC policy.  

% scalability 
\labeledfigurewide{fig-tpce-scalability}{Throughput when running TPCE-hybrid (left); TPCE-original (right).}
\labeledfigurewide{fig-tpcc-scalability}{Throughput when running TPCC-random (left); TPCC-fixed (right).}
In \figref{fig-tpce-scalability} (left), we fix contention degree to 20\% scan range and see scalability trends, increasing the number of workers. Overwhelmed by contentions, Silo does not achieve linear scalability, even with its excellent scalability of the underlying physical layer. ERMIA benefits from its robust CC scheme and its storage manager did not forestall achieving linear scalability. This figure shows that not only the scalability of underlying physical layer, but also CC scheme performance in logical level dictates overall performance. We also performed the same experiment in the original TPC-E, without the analytic transaction. As illustrated in \figref{fig-tpce-scalability} (right), both ERMIA-SI and Silo achieve linear scalability over 24 cores. Silo does not suffer from contention, since TPCE-original has insignificant contentions. ERMIA-SSI delivered 83\% of ERMIA-SI performance with 24 threads due to the serializability cost.

\subsection{TPC-C} 
\seclabel{eval:tpcc}
We also run TPC-C where lightweight OCC camp shine. This experiment will be focusing on evaluating storage manager's scalability naturally, as TPC-C imposes little pressure on CC scheme. 

% TPCC-fixed
We measure throughput of all systems in TPCC-fixed, varying the number of worker threads. \figref{fig-tpcc-scalability} (right) shows that all systems scale reasonably over 24 cores. Compared to Silo, ERMIA falls behind Silo by 10\% and 20\%, with SI and SSI respectively, at the peak performance for the following reasons. First, CC did not suppress throughput due to lack of contention. Second, Silo benefit from TPC-C's small cache footprint. Unlike Silo, ERMIA maintains multi-versions in the indirection array. It pays additional cache miss costs during traversing indirection array to find a visible version. This was invisible in the TPC-E because its large cache footprint supressed caching effect. \kk{perf result support this analysis; ERMIA has more cache misses from indirection array. but not very confident. } Thus, Silo outperforms ERMIA in workloads where 1) contention is rare and 2) transaction footprint is small enough to take advantage of cache-optimization; TPC-C is the compelling example of such workload. ERMIA-SSI had extra 10\% cost to guarantee serializability due to serial dependency checking.

% TPCC-random
We now run TPCC-random and see how the previous results change. As shown in the \figref{fig-tpcc-scalability} (right), we can see that Silo is more sensitive to contentions than ERMIA. Compared to TPCC-fixed, the throughput of Silo decreased by approximately 30\%, while ERMIA delivered 15\% lower performance, catching up with Silo.
%
%\subsection{Performance study}
%\seclabel{eval:perf-study}
%We performed factor analysis to figure out how much cost ERMIA pays for its critical components quantitatively. \figref{fig-tpcc-cycle} illustrates normalized CPU cycle breakdown in the TPCC-partitioned with 24 worker threads. As shown in the figure, the indirection array imposed 10\% extra cost which mainly came from last level cache misses; 32\% of cache misses occured in indirection array. In ERMIA-SSI, serializability check brought additional 10\% overhead.\kk{not sure about this number} In all systems, Masstree is the biggest bottleneck with more than 30\% out of total cycles. The overhead of log manager and total ordering were negligible(~4\%). This supports that fully-uncoordinated logging and abandonment of total order are not necessarily required to achieve scalability. Epoch-based resource managers also performed well without huge overhead(?\%)\kk{TODO. fill out overhead}.
