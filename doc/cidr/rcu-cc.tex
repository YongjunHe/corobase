% This is "sig-alternate.tex" V1.9 April 2009
% This file should be compiled with V2.4 of "sig-alternate.cls" April 2009
%
% This example file demonstrates the use of the 'sig-alternate.cls'
% V2.4 LaTeX2e document class file. It is for those submitting
% articles to ACM Conference Proceedings WHO DO NOT WISH TO
% STRICTLY ADHERE TO THE SIGS (PUBS-BOARD-ENDORSED) STYLE.
% The 'sig-alternate.cls' file will produce a similar-looking,
% albeit, 'tighter' paper resulting in, invariably, fewer pages.
%
% ----------------------------------------------------------------------------------------------------------------
% This .tex file (and associated .cls V2.4) produces:
%       1) The Permission Statement
%       2) The Conference (location) Info information
%       3) The Copyright Line with ACM data
%       4) NO page numbers
%
% as against the acm_proc_article-sp.cls file which
% DOES NOT produce 1) thru' 3) above.
%
% Using 'sig-alternate.cls' you have control, however, from within
% the source .tex file, over both the CopyrightYear
% (defaulted to 200X) and the ACM Copyright Data
% (defaulted to X-XXXXX-XX-X/XX/XX).
% e.g.
% \CopyrightYear{2007} will cause 2007 to appear in the copyright line.
% \crdata{0-12345-67-8/90/12} will cause 0-12345-67-8/90/12 to appear in the copyright line.
%
% ---------------------------------------------------------------------------------------------------------------
% This .tex source is an example which *does* use
% the .bib file (from which the .bbl file % is produced).
% REMEMBER HOWEVER: After having produced the .bbl file,
% and prior to final submission, you *NEED* to 'insert'
% your .bbl file into your source .tex file so as to provide
% ONE 'self-contained' source file.
%

%
% For tracking purposes - this is V1.9 - April 2009

\documentclass{sig-alternate}
%%\documentclass[letter]{sig-alternate}
%%\documentclass{vldb}

\usepackage{microtype}

\usepackage{ifthen} % \isempty macro..
\usepackage{graphicx}
\usepackage[usenames,dvipsnames]{color}
\usepackage[numbers,sort]{natbib}
\usepackage{verbatim}

% allow code listings with line numbering
\usepackage{listings}
%define how inlined source code should be displayed
\lstset {
language=C,
basicstyle=\scriptsize,
keywordstyle=\ttfamily\bfseries\color{blue},
commentstyle=\color{OliveGreen},
numbers=none,
numberstyle=\scriptsize,
numbersep=5pt,
tabsize=4,
gobble=4,
xleftmargin=20pt,
escapeinside={(@*}{*@)},
morekeywords={}
}

\newcounter{mycounter}

% by default footnotes are indented on the first line, which makes
% them look pretty ragged. Change them to be flush left with hanging
% indent (similar to references in the bibliography). NOTE: must
% include before hyperref package
\usepackage[hang,splitrule]{footmisc}
\setlength{\footnotemargin}{1em}

\def\thepapertitle{An architecture for fast and robust memory-optimized OLTP} 
\def\thepaperkeywords{Indirection array, Epoch-based, OLTP, Concurrency control}
  
% must import before hyperref (and algorithm, which imports it, must come after)
\usepackage{float} 
\usepackage[
  pageanchor=true,
  plainpages=false,
  pdfpagelabels,
  bookmarks,
  bookmarksnumbered,
  pdfborder=0 0 0,  %removes outlines around hyper links in online display
  colorlinks=true,
  linkcolor=blue,
  citecolor=Blue, % OliveGreen
  pdfpagelayout=TwoPageRight,
  pdftitle={\thepapertitle},
  pdfauthor={Ippokratis Pandis},
  pdfsubject={},
  pdfkeywords={\thepaperkeywords},
]{hyperref}

\floatstyle{ruled}
\usepackage{algorithm}
\newfloat{algorithm}{t}{lop}

% make floats waste way less space
% http://dcwww.camd.dtu.dk/~schiotz/comp/LatexTips/LatexTips.html
\renewcommand{\topfraction}{0.85}
\renewcommand{\textfraction}{0.1}
\renewcommand{\floatpagefraction}{0.80}

\setlength{\fboxsep}{1pt}
\let\oldcite\cite
\renewcommand{\cite}[1]{\colorbox{linkbg}{\oldcite{#1}}}
\definecolor{linkbg}{gray}{0.97}
\newcommand{\myref}[2]{\colorbox{linkbg}{\hyperref[#2]{%
      \ifthenelse{\equal{#1}{}}{}{#1 }\ref{#2}}}}

% make a really loud marker for citations I know I don't have yet
%% usage: \citeme
\makeatletter
\newcommand{\citeme}{\@ifstar
  \citemeStar%
  \citemeNoStar%
}
\makeatother

\newcommand{\citemeStar}{\citemeNoStar{citation needed}}
\newcommand{\citemeNoStar}[1]{
  {\color{Magenta}{\bf \fbox{citeme} #1}}
  % missing comment on line above is intentional
}

% TikZ ist kein Zeichenprogramm
\usepackage{tikz}
\usetikzlibrary{plotmarks}
\usetikzlibrary{arrows}
\usetikzlibrary{snakes}
\usetikzlibrary{shapes}
\usetikzlibrary{patterns}

% nice table formatting
\usepackage{booktabs}

% nice fractions, e.g., 1/4 and so forth...
\usepackage{nicefrac}

% flexible but compact enumerations
\usepackage{paralist}

\usepackage{subfigure}
\usepackage[noend]{algorithmic}
\usepackage{soul}
\usepackage{multirow}

%
% please place your own definitions here and don't use \def but
% \newcommand{}{}

\newcommand{\ippo}[1]{\noindent{\color{Red} {\bf \fbox{IP}} {\hl{\it#1}}}}
\newcommand{\ryan}[1]{\noindent{{\bf \fbox{Ryan}} {\hl{\it#1}}}}
\newcommand{\kk}[1]{\noindent{{\bf \fbox{KK}} {\hl{\it#1}}}}
\newcommand{\tianzheng}[1]{\noindent{{\bf \fbox{Tianzheng}} {\hl{\it#1}}}}

%% 1-column
% \labeledfigure{filename}[Short caption]{Long caption}
\newcommand{\labeledfigure}[1]{%
  \def\thelabeledfigure{#1}%
  \labeledfigurerelay%
}
\newcommand{\labeledfigurerelay}[2][]{%
  \begin{figure}[t]%
    \centering%
    \includegraphics{\thelabeledfigure}%
    \ifthenelse{\equal{#1}{}}%
               {\caption{#2}}% then
               {\caption[#1]{#2}}% else
               \label{fig:\thelabeledfigure}%
  \end{figure}%
}

%% 2-column
% \labeledfigure{filename}[Short caption]{Long caption}
\newcommand{\labeledfigurewide}[1]{%
  \def\thelabeledfigurewide{#1}%
  \labeledfigurewiderelay%
}
\newcommand{\labeledfigurewiderelay}[2][]{%
  \begin{figure*}[t]%
    \centering%
    \includegraphics{\thelabeledfigurewide}%
    \ifthenelse{\equal{#1}{}}%
               {\caption{#2}}% then
               {\caption[#1]{#2}}% else
               \label{fig:\thelabeledfigurewide}%
  \end{figure*}%
}

% this bad boy defines a pair of macros for handling chapter-aware
% cross references. If called like this:
% \makereftype{fig}{Fig.}{figure} it would define a macro \figlabel[1]
% (for labeling a figure) and \figref[3][\thechlabel][Fig.]  (for
% referring to it)
\newcommand{\makereftype}[2]{% command stem | default display
  \expandafter\newcommand\csname#1label\endcsname [1]{\label{#1:##1}}
  \expandafter\newcommand\csname#1ref\endcsname [1][]{%
    \expandafter\csname#1refrelay\endcsname%
  }%
  \expandafter\newcommand\csname#1refrelay\endcsname [2][#2]{%
      \myref{##1}{#1:##2}%
  }%
}

\makereftype{fig}{Figure}
\makereftype{sec}{Section}
\makereftype{tbl}{Table}
\makereftype{alg}{Algorithm}
\makereftype{line}{Line}

\makeatletter
\g@addto@macro\algorithm\scriptsize
\makeatother

\newenvironment{code}{\scriptsize}{}

% latex is terrible with widow/orphan lines. This is really
% heavy-handed but nothing else makes it pay attention.
\clubpenalty=10000
\widowpenalty=10000

% IP: Space saving tricks
%\usepackage[small,compact]{titlesec}
%\usepackage[small,it]{caption}
%\usepackage{times}

%% %% IP - HEADER/FOOTER
%% \lhead{}
%% \chead{\MakeUppercase{\bfseries *** please do not distribute ***}}
%% \rhead{}
%% \lfoot{}
%% \cfoot{\large{DRAFT, \today}}
%% \rfoot{}
%% %% IP - EOF HEADER/FOOTER


\begin{document}
%
% --- Author Metadata here ---
\conferenceinfo{7th Biennial Conference on Innovative Data Systems Research (CIDR'15)}{January 4-7, 2015, Asilomar, California, USA.}
%% \conferenceinfo{CIDR'13,}{Jan 6--9, Asilomar, CA, U.S.A.}
%% \CopyrightYear{2013}
%% \crdata{978-1-4503-1429-9/13/01}
% --- End of Author Metadata ---

\title{\thepapertitle}

\numberofauthors{1}
\author{
Kangnyeon Kim
~~
Tianzheng Wang
~~
Ryan Johnson
~~
Ippokratis Pandis$^\star$
\bigskip\\
$
\begin{array}{cc}
  \mbox{\affaddr{University of Toronto}} &
  \mbox{$^\star$\affaddr{Cloudera}}\\
  \email{{\large \sf {knkim,tzwang,ryan.johnson}@cs.utoronto.ca}} &
  \email{{\large \sf ippokratis@cloudera.com}}
\end{array}
$
}

\maketitle

%% Breaking the file into sections

% -*- tex-main-file:"rcu-cc.tex" -*-

\begin{abstract}

The emergence of systems with large main memories and massively parallel processors has triggered a resurgence of new high-performance transaction processing systems. 
Many of these systems adopt lightweight optimistic concurrency control schemes that are suitable only for a small fraction of transactional workloads.
We examine existing systems and argue that---for better or for worse---it is their system architecture that largely dictates the choice of concurrency control they employ.
Therefore, it is quite difficult, if not impossible, to adopt a different concurrency control scheme without significant changes to the rest of the system. 
The architecture also largely determines how the physical conflicts between concurrent threads are dealt, and how recovery is achieved.

We argue that transaction processing systems should be designed from the ground up with four basic requirements: to not heavily rely on partitioning; to provide flexible and robust concurrency control for the logical interactions between transactions; to address the physical interactions between threads in a scalable way; and to have a clear recovery methodology.  
We report on some early progress in designing a system from the ground up to support the requirements we lay out, and show how the resulting architecture achieves these goals without unnecessary sacrifices to performance in other areas.

\end{abstract}

%% -*- tex-main-file:"rcu-cc.tex" -*-

\section{Introduction}
\seclabel{intro}

%1. new hw trend has led to new systems
Modern systems with large main memories and massively parallel processors have inspired a new breed of high-performance memory-optimized OLTP systems \cite{Kallman+08,PandisJHA10,KemperN11,LarsonBDFPZ11,TuZKLM13}. These systems leverage spacious main memory to fit the whole working set in DRAM with streamlined, memory-friendly data structures. Further, optimizations for multicore and multi-socket hardware allow a much higher level of parallelism compared to conventional database systems. With disk overheads and delays removed, transaction latencies drop precipitously and worker threads can usually execute transactions to completion without interruption. The result is a welcome reduction in contention at the logical level and less pressure on whatever concurrency control (CC) scheme might be in place. A not very welcome result is an increasing pressure for scalable data structures and algorithms to cope with the increasing number of worker threads that concurrently execute transactions and need to communicate.

\labeledfigurewide{fig-write-ratio}{Performance of a memory-efficient OLTP engine with lightweight optimistic concurrency control, as the ratio of writes increases (left); and as the size of the database decreases (right).}

\vspace{2mm}
{\bf Interactions at the logical level.} 
%2. how current workloads make cc important again.
Many designs exploit the reduction in the pressure on CC, by employing very optimistic and lightweight schemes, boosting even further the performance of these systems on suitable workloads.
But, as is usually the case, it appears that database workloads stand ready to absorb any and all concurrency gains the memory-optimized systems have to offer. In particular, there is high demand for database systems that can handily serve database workloads that evolve increasingly heterogeneous, blending the gap between transaction and analytical processing. This trend is at least partly enabled by the improved concurrency and reduced contention offered by memory-optimized systems \cite{Farber+12}. Mixed workloads have two significant impacts on CC, however. First, the write/read ratio decreases from 1:2 (e.g. TPC-C) to 1:10 or less (e.g. TPC-E \cite{Chen+10,TozunPKJA13}), usually {\it by increasing the number of reads as the number of writes remains stable}. 
Second, workloads frequently include some fraction of large transactions that are {\it read-mostly rather than read-only}---a trend reflected in the TPC-E benchmark. Unfortunately, both of these workload properties result to an increase in effective concurrency control footprints, adding pressure to the CC scheme. 
Therefore, going forward and as the industry shifts to heterogeneous workloads served by memory-optimized engines, it is vital for them to employ effective and robust CC schemes. 

%3. current schemes: 2PL blocks + deadlock issues. existing schemes in new systems (mostly OCC) suffer (1) long tx with other tx going on can't commit (2) write clobber read
We observe that the CC schemes currently in vogue with memory-optimized system are not robust under contention, particularly when short write-intensive transactions coexist with longer read-mostly transactions.
For example, the two main families of approaches can be loosely classified as two-phase locking (2PL) and optimistic concurrency control (OCC). 2PL is common in traditional disk-oriented systems, and is often criticized because of high overheads, its policy of blocking transactions (leading to deadlocks and other scheduling problems), and a tendency to ``lock up'' (performance crash) once the aggregate transactional footprint grows too large, a state quickly attained when large transactions enter the system. OCC, on the other hand, never blocks readers---and may not even block writers---thus avoiding most scheduling issues. Although they differ in details, the rising generation of memory-optimized systems almost universally adopts a form of OCC that is effectively single-versioned, with read footprint validation at pre-commit.  Two systems that characteristically employ this type of OCC are Microsoft's Hekaton \cite{LarsonBDFPZ11} and Silo \cite{TuZKLM13}. This type of approach suffers badly in high-parallelism systems \cite{YuBPDS14} because transactions must abort if any portion of their read footprint is overwritten before they commit. 
In \figref{fig-write-ratio} we demonstrate how the performance of Silo, a representative of the camp of transaction processing engines with lightweight OCC, degrades as transactions have larger read footprint or when contention increases. (\secref{eval:setup} has details about the experimental setup.) \figref{fig-write-ratio}(left) shows that it just takes 0.1\% or 1\% of the touched records to be updates for the transaction throughput to drastically drop. While \figref{fig-write-ratio}(right) shows that the abort rate grows quickly as the same number of threads operate on smaller TPC-C databases, thereby on higher contention.

\vspace{2mm}
{\bf Interactions at the physical level.} 
As commodity server hardware becomes increasingly parallel~\footnote{Note that the upcoming generation of Intel server-grade processor, Haswell-EP, comes with up to 18 cores (and 36 hyperthreads) per socket.} many of the low-level issues (latching, thread scheduling, etc..) and design decisions---at the architecture level---need to be revisited. The form of logging used, the storage management architecture, and scheduling policies for worker threads can impose drastic constraints on which forms of CC can be implemented at all, let alone efficiently. 
Therefore, it is difficult or impossible to adopt a different CC scheme without significant changes to the rest of the system. 
For example, it was reported in \cite{PortsG12} that the implementation effort require to add support for SSI in Postgres was very high. 
The point is not that such design choices should be avoided, but rather that they should be made only with a full awareness of the consequences for concurrency control. 

\vspace{2mm}
{\bf Partitioning.} Some systems sidestep the issues of logical and physical contention as well as the accompanying implementation complexity entirely by adopting physical partitioning and a single-threaded transaction execution model \cite{Kallman+08,KemperN11}. But that introduces a different set problems for mixed workloads and for workloads that are inherently difficult to partition.  Given the developments in scaling-out the performance of distributed OLTP systems, especially for easy-to-partition workloads, e.g. \cite{Corbett+12,BailisFHGS14,ThomsonA10}, as well as for high availability and cost-effectiveness reasons, we predict that the successful architectures will combine scale-out solutions build on top of non-partitioning-based scale-up engines within each node.

\vspace{2mm}
{\bf ERMIA.} 
In \secref{desired} we are laying out the design principles that we believe are critical for transaction processing engines in the environment of highly-parallel servers with ample main memory. Next, on \secref{design}, we are presenting {\em ERMIA}, a memory-optimized transaction processing architecture that by combining epoch-based resource management and the indirection array technique \cite{SadoghiRCB13}, provides more robust CC, scalable thread interactions and easy recovery.  
\secref{eval} compares the performance of an ERMIA prototype against a representative of the new breed of memory-optimized shared-everything transaction processing systems, and shows how the resulting architecture does not necessarily sacrifice performance in other areas.


\section{Design directions}
\seclabel{desired}

In this section we briefly discuss our desired properties of a transaction processing system architecture. We primarily focus on three areas: the concurrency control mechanism that determines the interaction between concurrent transactions at the logical level; the mechanism that controls the interaction/communication of threads at the physical level; and recovery. As we already argued in \secref{intro}, we are aiming for a scalable single-node design that relies as little as possible to physical partitioning.  

\vspace{2mm} 
{\bf Concurrency control:} 
Broadly speaking, there are two camps of CC methods: the pessimistic, e.g. two-phase locking (2PL), and the optimistic (OCC). As it has been shown in the past, e.g. \cite{AgrawalCL87}, in theory in presence of contention the  pessimistic methods beat optimistic if the overhead of those pessimistic methods is comparable with the overhead of the optimistic counterparts. However this is not easy to achieve. For example, a study of the SHORE storage manager estimates that there is at a 25\% overhead for locking-based pessimistic methods \cite{HarizopoulosAMS08}. That's is a lot of slack for OCC to outperform pessimistic.

Having said that, typical memory-optimized engines that employ lightweight OCC and running on modern commodity servers, already provide quite high performance. Especially in the workloads they are optimized for (short-running partitionable transactions with small read and write footprints). Given that pessimistic CC is way more robust, the designer may seriously consider taking the hit and losing some of the peak performance in order to provide more robust behavior. In other words, it may be ok to lose say 15-20\% of the peak performance if that measures in millions or hundreds of thousands of transactions per sec.

Ideally the CC mechanism should not only have a low false positive in rate detecting conflicts, but it should allow the system to detect the glaring conflict cases (cases where a transaction is destined to fail) as early enough and not during pre-commit.
There are different flavor of optimistic, or opportunistic, CC. Many recent systems adopt a lightweight validation step at the end of the transaction, during pre-commit. Such kind of validation is very opportunistic, it is not robust, leaving the system vulnerable in many workloads.
Also, if a conflict is detected, either at pre-commit, or hopefully earlier, then blind retries without any guarantee about the success of the transaction at this time around waste useful cycles and create problems \cite{PortsG12}. In other words, the system should avoid repeatedly hitting the same conflicts. Instead, safe retries are desirable. 

At least to our knowledge, there are no (publicly available) systems, that are both fast enough and have the appropriate infrastructure to support the implementation of proper, robust CC schemes. The infrastructure matters terribly. It decides whether it is even possible to implement a particular CC scheme, and whether that would be practical. For example, the effort to enhance Postgres with serializable snapshot isolation (SSI) required a very large implementation effort, since the team had to integrate what it is essentially a lock manager~\footnote{    Once all this groundwork was done, extending Postgres to other CCs was relatively easy.}. And even then, the achieved performance was not impressive.  Many of the design decisions described in \secref{design} were specifically taken in order not to limit the implementation of CC schemes in some way.

\vspace{2mm}
{\bf Interactions at the physical level:} 
The interactions at the physical level are typically handled by a low-level component of any transaction processing system called the storage manager. 
One very promising technique for storage managers that provides desirable properties is the indirection array, as for example it is presented in \cite{SadoghiRCB13} and also used by Hekaton \cite{Diaconu+13}.
For example, with an indirection array there is less chatter into the log. In a record update only the corresponding entry in the array needs to be updated. Otherwise, an update may result cascading updates that may propagate up to the root. Additionally, on record updates the system does not have to update every reference to that record, say from secondary indexes. 

As a comparison, Silo does not employ indirection arrays but instead it performs in place updates. That is, for all practical purposes in Silo there is effectively a single committed version of an object with perhaps a private copy. So is Hekaton. Both systems are multi-versioned only for read-only transactions.

The indirection arrays are suitable for the physical implementation of CC for multi-versioned systems, as a single compare-and-swap (CAS) operation installs a new version of an object. 
With indirection arrays even the anti-caching technique \cite{DeBrabantPTSZ13} is largely simplified. That is, in the vanilla anti-caching algorithm all the secondary indexes have to be updated, whenever a record is evicted from memory with some considerable overhead. With the indirection array there is no need for more than one update. 
In some sense the indirection array is a lightweight bufferpool. 

There are also some sort of esoteric reasons, detailed in \secref{design:oid}, for using indirection arrays: space management becomes easier, as it becomes easier to implement cache-friendly compact index structures, such as CSBs \cite{RaoR00}, as long as we can tolerate the extra level of indirection. That is, since for the aforementioned reasons we are inclined to employ the indirection array, those benefits  come from free. 

The flexibility in implementing CC schemes is greatly enhanced is we can establish total orders. Therefore we prefer the system to have a centralized log, which can be used for establishing the transaction commit order. 
In contrast, Silo employs a epoch-based ordering that is only partial. 

{\bf Recovery:} 
-- Recovery has also to be a first class citizen
- having a log that requires just one central communication, through a CAS
- lots of hand-wavvy claims
- you need to have to recovery story shorted out before you start
- voltdb -- does log shipping
- silo -- does not even implement recovery
- hekaton -- hand-wavvy description


The combination of indirection arrays and centralized logging drastically simplifies recovery. There is no need for undo at all, redo is trivial just need log analysis pass to restore in-memory data structures. That is, need checkpoint plus log analysis to recover!


%% -*- tex-main-file:"rcu-cc.tex" -*-

\section{Fast main-memory OLTP with rcu-based concurency control}
\seclabel{design}

In tis section we describe in more detail several key pieces of the system we are developing, with a focus on why we choose the design trade-offs we do.

\subsection{Log manager}

The log manager is a pivotal piece of most database engines. It provides---if anything does---a centralized point of coordination that other pieces of the system build off of and depend on. We have developed a logging scheme that  generates a commit LSN for the transaction and reserves buffer space for the transaction's log records with a single global atomic operation. Achieving this required two key insights: first, the transaction can combine its log records into large blocks, avoiding the redundancy of writing individual log record headers and reducing the number of trips to the log. Second, the LSN space need not be contiguous as long as we can still convert easily from LSN to disk address and back.

The first property arises from our use of append-only storage. We achieve the second property by assigning each LSN to a ``segment'' and storing its segment number in the low order bits; the LSN's position on disk can be determined by looking up, and subtracting off, its segment's starting offset (read-only). Transactions race to ``open'' a new segment if they obtain an LSN past the end of the current one, and unlucky transactions holding a LSN in the gap between two segments simply discard it and request a new one. Segments can be 100GB or more in size, however, so overflows are quite rare. Once an LSN and segment have been assigned, the transaction verifies availability of space in the log's circular memory buffer (again, read-only); only in case the buffer is full will transactions have to block pending space, but high-end disk arrays readily absorb the fully sequential write-only I/O stream.

An additional feature of the log is that transactions acquire a commit LSN before entering pre-commit, allowing validation of multiple transactions to proceed smoothly in parallel; depending on the outcome of pre-commit, a transaction either writes its log entries or a skip record aborts to the reserved log block.

Finally, because the shared counter is implemented as a wait-free linked list, the transaction can notify the log writer that its block is ready to flush by simply flagging its node as ``dead'' (a blind store). The log writer periodically scans the list and writes out all log blocks that precede the oldest ``live'' buffer allocation; transactions do not touch each others' nodes, and the log writer only reads them and flags dead nodes for the garbage collector.

\subsection{Epoch-based resource management}
We have developed a lightweight epoch management system that can track multiple timelines (of differing granularities) in parallel. A multi-transaction-scale epoch manager implements garbage collection of dead versions and deleted records, a medium-scale epoch implements a read-copy-update (RCU) mechanism that manages physical memory and data structure usage\cite{RCU}, and a very short timescale epoch manager tracks transaction IDS, which we recycle aggressively (see below).

The key to efficiency here is to avoid flagging stragglers unless it is absolutely necessary (because coordinating with non-responsive threads is very expensive). To avoid this, the system does not attempt to reclaim resource for epoch $N$ until epoch $N+2$ begins. This way, potential stragglers have all of epoch $N+1$ to quiesce without penalty; however, epoch $N+3$ cannot begin until the last straggler from epoch $N$ completes. This four-phase scheme communicates far less with stragglers than the traditional two-phase approach while maintaining the same worst-case timing bound. It allows us to track epochs at a very fine granularity when necessary. 

\subsection{Transaction management}

Each transactions in the system is assigned a slot in a global transaction state table when it begins. This fixed-size table holds its begin time (which is the log's end LSN at the time it started), status, and end time (if applicable). Transaction ids are a combination of table offset and epoch, with an epoch manager to prevent entries from being recycled too soon. Update transactions write their XID into each version they create, change their status to pre-commit, acquire a commit LSN (or are given one by an impatient peer), and finally commit atomically by changing their status to ``committed.'' A post-commit cleanup step involves replacing XID stamps with the transaction's commit LSN, at which point the state table entry is no longer needed can be recycled by the epoch manager. Other transactions that encounter an XID in a version can reliably verify its commit status and age by visiting the XID table, and---if necessary---will help a peer enter pre-commit by racing to acquire a log block on its behalf. 

\subsection{Indirection arrays}
\seclabel{design:oid}

The indirection arrays used in ERMIA are very similar to the ones proposed in the literature. In short, all logical objects are identified by an object ID (OID) that maps to a slot in an OID array that contains the physical pointer. The pointer may reference disk, or a chain of versions stored in memory. As with Hekaton, uncommitted versions are never written to disk, but unlike Hekaton, we dispense with delta records (too expensive to apply) and use pure copy-on-write. New versions can be installed by an atomic compare-and-swap operation, and an uncommitted record at the head of the chain constitutes a write lock for CC schemes that care to track W-W conflicts (as most do). 

\ippo{Perhaps come up with a name.}

\subsection{Concurrency control}
\seclabel{design:cc}

The system has been designed from the ground up to allow efficient implementations of a variety of concurrency control mechanisms. It can use read set validation (like SILO and Hekaton), but can also provide snapshot isolation to writers and even efficient implementations of serializable snapshot isolation that have been proposed recently\cite{Fekete}. We are also in the process of developing new CC schemes which promise lower abort rates than SSI with lower overhead and reduced implementation complexity. The other components in the system work together to make efficent CC possible: efficient (but optional) multi-versioning allowed by the indirection arrays, the total commit ordering afforded by the log, and the ability to determine easily the age of a version thanks to the transaction manager. Only the first (MVCC via indirection array) is available in other systems, and the other two features are key enablers of the new CC schemes we are developing.

\subsection{Recovery}
\seclabel{design:recovery}

Recovery is straightforward because the log contains only committed work. OID arrays are the only real source of complexity, as they are volatile in-memory structures that make it possible to find all other objects in the system. It turns out that logical objects (records) are physically logged, while physical data (allocator state and OID array contents) use logical logging.  OID arrays are themselves objects stored in a master OID array, but they are updated in place to avoid overloading the log, with changes replayed by a log analysis step that reads only log block headers. This analysis step is very fast, because the skipped-over log payloads account for 90\% or more of the total log. In order to support efficient recovery, system transactions occasionally checkpoint the OID arrays using a fuzzy checkpointing mechanism to minimize the impact on user transactions. Because the log is the database, recovery only needs to rebuild the OID arrays in memory; anti-caching will take care of loading the actual data, though background pre-loading is highly recommended to minimize cold start effects.

\subsection{Prototype Implementation}
\seclabel{design:prototype}

We implement a prototype of the ERMIA architecture and measure its performance.  For the implementation of the prototype we use a large fraction of the publicly available Silo codebase~\footnote{Silo's codebase can be downloaded from: https://github.com/stephentu/silo.} Silo uses the Masstree \cite{MaoKM12} as a cache-efficient index structure. 

%% -*- tex-main-file:"rcu-cc.tex" -*-

\section{Evaluation}
\seclabel{eval}

The purpose of this evaluation section is twofold.  First, we want to show that the recent proposals for main-memory OLTP solutions with light-weight optimistic concurrency control schemes, such as Silo, can perform well and are suitable for only on a limited subset of transactional applications, primarily due to limitations imposed by their (tightly-integrated) concurrency control mechanism.  Second, that the proposed RIA design offers a more efficient concurrency control, which allows its performance to remain high for a larger spectrum of workloads.  At the same time, RIA does not suffer significant sacrifices in performance, even in workloads where the specialized solutions shine.  We are not attempting a detailed evaluation, as this would be more suitable for a longer document.

\subsection{Experimental setup}
\seclabel{eval:setup}

We use a 4-socket server with 6-core Intel Xeon E7-4807 processors, for a total of 24 physical cores and 48 hyperthreads. The machine has 64GB of RAM. 
For our comparison we use Silo.  Silo's SLAB allocator is given 32GB of memory and its logging is disabled. We use the TPC-C database with 32 Warehouses (scaling factor of 32). We use the Stock table for the transactions.  The Stock table has 100K*Warehouses (3.2M) records. Each transaction randomly peaks a number of records to read and a smaller fraction of records to update.  We run each experiment for 60secs.

\labeledfigurewide{fig-rand-100}{Rate of committed and aborted transactions for Silo as the number of concurrent threads increases, for an increasing ratio of writes. Each transaction performs 100K random reads.}

\labeledfigurewide{fig-rand-300}{Rate of committed and aborted transactions for Silo as the number of concurrent threads increases, for an increasing ratio of writes. Each transaction performs 300K random reads.}

\subsection{Performance as contention increases}
\seclabel{eval:contention}

\figref{fig-rand-100} shows the performance of Silo as the number of concurrent threads increases, for an increasing ratio of writes, when each transaction performs 100K random reads.

Things get even worse as we increase the read footprint of the transactions. For example, \figref{fig-rand-300} shows the performance of Silo as the number of concurrent threads increases, for an increasing ratio of writes, when each transaction is three times larger, performing 300K random reads.





%% -*- tex-main-file:"rcu-cc.tex" -*-

\section{Related work}
\seclabel{related}

In terms of concurrency control, one of the most important studies has been \cite{AgrawalCL87}, where it was shown that if the overhead of pessimistic two-phase locking can be comparable to the overhead of optimistic methods then the pessimistic one is superior. The same study showed that it is beneficial to abort transactions that are going to abort as soon as possible. We corroborate these findings. 
Serializable snapshot isolation, for example as this is implemented in Postgres \cite{PortsG12}. 
 
The indirection map, which is central to RIA's design, is a well-known technique, for example presented in \cite{SadoghiRCB13}.

With the continuous increase in size of main memories and decrease of their cost as well as the emergence of multi-core and multi-socket hardware, there has been a resurgence of research in the area of in-memory transaction processing systems.  
To deal with logical conflicts between transactions, many of these systems adopt lightweight optimistic concurrency control schemes that are suitable only for a small fraction of transactional workloads.
The designs can be categorized in three categories: non-partitioning- and partitioning-based systems and clustered solutions. 

%% \ippo{Non-partitioning}
Silo's  \cite{TuZKLM13} employs a light-weight optimistic concurrency control that, as we showed, performs well only in a limited set of workloads. 
Hekaton \cite{Diaconu+13} is a memory-optimized transaction processing system by Microsoft. Hekaton employs a multi-versioning concurrency control \cite{LarsonBDFPZ11}, similar to Silo's. It is worth mentioning that Hekaton also uses a technique similar to the indirection map, which we also use. 

\ryan{Any more on concurrency control?}

%% \ippo{Partitioning-based}
H-Store (and its commercial version, VoltDB) is a characteristic partitioning-based system \cite{Kallman+08}. H-Store physically partitions each database to as many instances as the number of available processors, and each processor executes each transaction in serial order without interruption.  
Problems raise when the system has to execute mutli-site transactions, transactions that touch data from two or more separate database instances. Lots of work has been put in the area, including low overhead concurrency control mechanisms \cite{JonesAM10}, but also partitioning advisors that help to co-locate data that are frequently accessed in the same transactions, thereby reducing the frequency of multi-site transactions, e.g. \cite{CurinoJZM10,PavloJZ11,TranNST14}.
Hyper \cite{KemperN11} follows H-Store's single-threaded execution principle.  To scale up to multi-cores they employ the hardware transactional memory capabilities of the latest generation of Intel (Haswell) and Power (P8) processors \cite{LeisKN14}. 

DORA \cite{PandisJHA10}  employs logical partitioning 
PLP \cite{PandisTJA11} extends the data-oriented execution principle, by employing physiological partitioning. Under PLP the logical partitioning is reflected at the root level of the B+tree indexes that now are essentially multi-rooted. PLP is
Both DORA and PLP use Shore-MT's codebase \cite{JohnsonPHAF09}, which is a scalable but disk-optimized storage manager. Hence, their performance lacks in comparison with the memory-optimized proposals. Additionally, even though only logical, there is a certain overhead in the performance due to the partitioning mechanism employed.  

%% \ippo{Cluster solutions}
In addition to the work on scaling up the performance of transaction processing systems in mutlicore and multisocket environment, there has been also lots of interest on scaling out. Those scale out systems, such as Google's Spanner \cite{Corbett+12}, RAMP \cite{BailisFHGS14} and Calvin \cite{ThomsonA10}, emphasize the weakness of the partitioning-based camp. Because the easy to partition workloads and databases would have already been partitioned of different physical nodes. Therefore whatever data are assigned to a single node it would be quite difficult to further partition.  Hence the need of scalable multicore and multisocket transaction processing system designs.

%% -*- tex-main-file:"rcu-cc.tex" -*-

\section{Conclusion }
\seclabel{conclusion}

In this paper we underlined the weaknesses of recent MVCC-based transaction processing system proposals, and presented a novel system with much more robust performance due to its concurrency control.
The point we want to make is that concurrency control is a fundamental component of any transaction processing system, and that it cannot be baked in after the fact to system. Instead, transaction processing system designers should think of concurrency control from the beginning as the decision about the concurrency control mechanism will dictate most of the design decision of all the other components of the systems. 



%%\input{rcu-cc-acks}

%% IP: We should paste the .bbl here for the camera-ready
\bibliographystyle{abbrv}
%% \bibliographystyle{acm-custom}
\setlength{\bibsep}{4pt}
{\small  %\scriptsize % no need for big print here...
\bibliography{../common/biblio}
}

%\balancecolumns % GM June 2007
% That's all folks!
\end{document}
