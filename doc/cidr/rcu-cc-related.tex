%% -*- tex-main-file:"rcu-cc.tex" -*-

\section{Related work}
\seclabel{related}

In terms of concurrency control, one of the most important studies has been \cite{AgrawalCL87}, where it was shown that if the overhead of pessimistic two-phase locking can be comparable to the overhead of optimistic methods then the pessimistic one is superior. The same study showed that it is beneficial to abort transactions that are going to abort as soon as possible. We corroborate these findings. 
Serializable snapshot isolation, for example as this is implemented in Postgres \cite{PortsG12}. 
 
The indirection map, which is central to RIA's design, is a well-known technique, for example presented in \cite{SadoghiRCB13}.

With the continuous increase in size of main memories and decrease of their cost as well as the emergence of multi-core and multi-socket hardware, there has been a resurgence of research in the area of in-memory transaction processing systems.  
To deal with logical conflicts between transactions, many of these systems adopt lightweight optimistic concurrency control schemes that are suitable only for a small fraction of transactional workloads.
The designs can be categorized in three categories: non-partitioning- and partitioning-based systems and clustered solutions. 

%% \ippo{Non-partitioning}
Silo's  \cite{TuZKLM13} employs a light-weight optimistic concurrency control that, as we showed, performs well only in a limited set of workloads. 
Hekaton \cite{Diaconu+13} is a memory-optimized transaction processing system by Microsoft. Hekaton employs a multi-versioning concurrency control \cite{LarsonBDFPZ11}, similar to Silo's. It is worth mentioning that Hekaton also uses a technique similar to the indirection map, which we also use. 

\ryan{Any more on concurrency control?}

%% \ippo{Partitioning-based}
H-Store (and its commercial version, VoltDB) is a characteristic partitioning-based system \cite{Kallman+08}. H-Store physically partitions each database to as many instances as the number of available processors, and each processor executes each transaction in serial order without interruption.  
Problems raise when the system has to execute mutli-site transactions, transactions that touch data from two or more separate database instances. Lots of work has been put in the area, including low overhead concurrency control mechanisms \cite{JonesAM10}, but also partitioning advisors that help to co-locate data that are frequently accessed in the same transactions, thereby reducing the frequency of multi-site transactions, e.g. \cite{CurinoJZM10,PavloJZ11,TranNST14}.
Hyper \cite{KemperN11} follows H-Store's single-threaded execution principle.  To scale up to multi-cores they employ the hardware transactional memory capabilities of the latest generation of Intel (Haswell) and Power (P8) processors \cite{LeisKN14}. 

DORA \cite{PandisJHA10}  employs logical partitioning 
PLP \cite{PandisTJA11} extends the data-oriented execution principle, by employing physiological partitioning. Under PLP the logical partitioning is reflected at the root level of the B+tree indexes that now are essentially multi-rooted. PLP is
Both DORA and PLP use Shore-MT's codebase \cite{JohnsonPHAF09}, which is a scalable but disk-optimized storage manager. Hence, their performance lacks in comparison with the memory-optimized proposals. Additionally, even though only logical, there is a certain overhead in the performance due to the partitioning mechanism employed.  

%% \ippo{Cluster solutions}
In addition to the work on scaling up the performance of transaction processing systems in mutlicore and multisocket environment, there has been also lots of interest on scaling out. Those scale out systems, such as Google's Spanner \cite{Corbett+12}, RAMP \cite{BailisFHGS14} and Calvin \cite{ThomsonA10}, emphasize the weakness of the partitioning-based camp. Because the easy to partition workloads and databases would have already been partitioned of different physical nodes. Therefore whatever data are assigned to a single node it would be quite difficult to further partition.  Hence the need of scalable multicore and multisocket transaction processing system designs.
