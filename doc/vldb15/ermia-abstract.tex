% -*- tex-main-file:"rcu-cc.tex" -*-

\begin{abstract}

Large main memories and massively parallel processors have triggered not only a resurgence of new high-performance transaction processing systems, but also the evolving of heterogeneous read-mostly transactions. However, many of these systems adopt lightweight optimistic concurrency control schemes that are only suitable for short, write-intensive workloads, similar to the TPC-C benchmark. By analyzing the desired features of main-memory optimized database systems, we argue that it is the system architecture that largely dictates the concurrency control scheme a system can employ, hence the type of workloads that can be gracefully handled.
Therefore, it becomes difficult, if not impossible, to adopt a different concurrency control scheme to robustly handle various workloads without significant changes to the rest of the system. 

We believe that transaction processing systems should be designed from the ground up with four basic requirements: to not heavily rely on partitioning; to provide flexible and robust concurrency control for the logical interactions between transactions; to address the physical interactions between threads in a scalable way; and to have a clear recovery methodology.  
We report on the design and prototype implementation of a system, called ERMIA, from the ground up to support the requirements we lay out. 
Our evaluation  shows how the resulting architecture achieves these goals without unnecessary sacrifices to performance in other areas. In particular, we show that ERMIA achieves comparable performance with a recent high-performing transaction processing system in TPC-C, which is the workload most of the recent systems are optimized for. On the other hand, for a modified TPC-E workload with an extra read-mostly transaction, ERMIA achieves up to 40\% higher performance.  

\end{abstract}
